\documentclass[aspectratio=169]{beamer}
\usepackage{xspace}
\usepackage{tikz}
\usepackage{url,color,minted}
\usepackage{pgf,pgflibraryshapes}
\usetikzlibrary{shadows,arrows}

\usepackage[siunitx]{circuitikz}

\tikzstyle{linedot} = [draw, thick, color=black!50, dotted]
\tikzstyle{linepart} = [draw, thick, color=black!50, dashed]
\tikzstyle{linegrey} = [draw, thick, color=black!50]
\tikzstyle{linesolid} = [draw, thick, color=black, -latex', align=center]

\tikzstyle{ellipsesolid} = [draw, ultra thick, color=blue, ellipse, -latex', align=center]



\graphicspath{{./fig/},{../circuit/}}

\mode<presentation> { % handout / presentation
	\usetheme{Darmstadt}
	\setbeamertemplate{navigation symbols}{}
	\defbeamertemplate*{footline}{infolines2 theme}{
	  \leavevmode%
	  \hbox{%
	  \begin{beamercolorbox}[wd=\paperwidth,ht=2.25ex,dp=1ex,right]{quaternary}%	 
	    \insertframenumber{} / \inserttotalframenumber\hspace*{1ex} 
	  \end{beamercolorbox}}%
	  \vskip0pt%
	}
  
  \setbeamercolor*{palette primary}{use=structure,fg=black,bg=structure.fg!40!white}
  \setbeamercolor*{palette secondary}{use=structure,fg=black,bg=structure.fg!60!white}
  \setbeamercolor*{palette tertiary}{use=structure,fg=black,bg=structure.fg!90!white}
  \setbeamercolor*{palette quaternary}{fg=white,bg=black}

  \setbeamercolor*{sidebar}{use=structure,bg=structure.fg}

  \setbeamercolor*{palette sidebar primary}{use=structure,fg=structure.fg!10}
  \setbeamercolor*{palette sidebar secondary}{fg=white}
  \setbeamercolor*{palette sidebar tertiary}{use=structure,fg=structure.fg!50}
  \setbeamercolor*{palette sidebar quaternary}{fg=white}

  \setbeamercolor*{titlelike}{parent=palette primary}

  \setbeamercolor*{separation line}{}
  \setbeamercolor*{fine separation line}{}
}

\pgfdeclareimage[height=1cm]{logoRIOT}{fig/RIOT_logo.png}
\pgfdeclareimage[height=1cm]{logoSTM32}{fig/STM32_logo.png}

\title[Lecture 1]{How to use a Digital Sensor with RIOT}
\subtitle{using an STM32 Nucleo-64 F401RE development board}

\author[I.Chatzigiannakis]{Ioannis Chatzigiannakis}

\institute{\url{https://github.com/ichatz/riotos-apps}}

\date{}

\logo{\pgfuseimage{logoRIOT}\pgfuseimage{logoSTM32}}

\begin{document}

{
\setbeamercolor{upper separation line head}{fg=white, bg=white}
\setbeamercolor{lower separation line head}{fg=white, bg=white}
\setbeamercolor{title in head/foot}{fg=white, bg=white}
\setbeamercolor{institute in head/foot}{fg=white, bg=white}
\setbeamercolor{date in head/foot}{fg=white, bg=white}
\setbeamercolor{author in head/foot}{fg=white, bg=white}
\setbeamercolor{date in head/foot}{fg=white, bg=white}
\setbeamercolor{section in head/foot}{fg=white, bg=white}
\setbeamercolor{subsection in head/foot}{fg=white, bg=white}
\setbeamercolor{subsubsection in head/foot}{fg=white, bg=white}
\setbeamercolor{quaternary}{fg=white, bg=white}
\setbeamertemplate{headline}{}

\frame{\titlepage}

}

\section{How to use a Digital Sensor with RIOT}

\subsection{}

%_____________________________________________________________
\begin{frame}{}

\begin{block}{}
Low-power measurement of the ambient temperature and humidity with an \href{https://www.st.com/en/evaluation-tools/nucleo-f401re.html}{STM32 Nucleo-64 F401RE development board} and the \href{https://github.com/RIOT-OS/RIOT}{RIOT operating system}.
\end{block}

\bigskip

Required hardware components:

\begin{itemize}
	
\item STM32 Nucleo-64 F401RE

\item DHT22 digital sensor

\item 3 Female to male jumper wires

\end{itemize}
\end{frame}


%_____________________________________________________________
\begin{frame}{}
\begin{columns}

\column{.75\textwidth}
\begin{itemize}

\item The DHT22 digital sensor is a widely diffused component

\begin{itemize}

\item Measures temperature and relative humidity

\item Uses a custom protocol which use a single wire/bus for communication. 

\end{itemize}

\item The DATA wire used for communication between STM32 MCU and the DHT22.

\begin{itemize}

\item A 4.7K or 10K pull-up resistor is used to bring the bus in an IDLE state when there is no communication taking place.

\item  A continuous HIGH on the line denote an IDLE state.

\item The STM32 MCU acts as the bus controller and hence is responsible for initiating communication (i.e., read).

\end{itemize}

\end{itemize}

\column{.25\textwidth}
\centering
\includegraphics[width=\textwidth]{dht22}
\vspace{1cm}

\end{columns}
\end{frame}


%_____________________________________________________________
\begin{frame}{DHT22 Communication Protocol}
\begin{enumerate}

\item The STM32 MCU pulls it to a LOW for 18ms and HIGH for around 20 to 40us.
\item DHT22 detects a START and responds by pulling the line LOW for 80$\mu$s.
\item DHT22 pulls it HIGH for 80$\mu$s which indicates that it is ready to send 40 bits  data.
\item Each bit starts with a 50$\mu$s LOW followed by 26-28$\mu$s for a ``0'' or 70$\mu$s for a ``1''.
\item To ends, the Line is pulled HIGH by the pull-up resistor and enters IDLE state.

\end{enumerate}

\centering
\includegraphics[width=.65\textwidth]{dht22_protocol}
\end{frame}


%_____________________________________________________________
\begin{frame}{DHT22 Data Format}
\begin{enumerate}

\item 1st Byte: Relative Humidity Integral Data in \% (Integer Part)
\item 2nd Byte: Relative Humidity Decimal Data in \% (Fractional Part) – Zero for DHT11
\item 3rd Byte: Temperature Integral in Degree Celsius (Integer Part)
\item 4th Byte: Temperature in Decimal Data in \% (Fractional Part) – Zero for DHT11
\item 5th Byte: Checksum (Last 8 bits of {1st Byte + 2nd Byte + 3rd Byte+ 4th Byte})

\end{enumerate}

\centering
\includegraphics[width=.65\textwidth]{dht22_data_format}
\end{frame}



%_____________________________________________________________
\begin{frame}{}
\hspace*{1cm}
\includegraphics[width=.65\textwidth]{circuit_bb}
\end{frame}




%=============================================================
\subsection{The RIOT operating system}
\logo{\pgfuseimage{logoRIOT}}




%_____________________________________________________________
\begin{frame}[fragile]{Hardware Independent elements}
\vspace{-.5cm}
\begin{columns}

\column{.5\textwidth}
\begin{exampleblock}{Makefile}
\begin{minted}{c}
# name of application
APPLICATION = photocell

# Path to the RIOT base directory:
RIOTBASE ?= $(CURDIR)/../../RIOT

# Modules to include:
USEMODULE += dht
USEMODULE += fmt	

# RIOT features
FEATURES_OPTIONAL += periph_rtc
\end{minted}
\end{exampleblock}
%$

\column{.4\textwidth}

\vspace{4.45cm}

We wish to use the \alert{DHT module}.\\
\alert{FMT} = Format Module to help with the DHT protocol.

\bigskip

The \alert{RTC} module used for power management.

\column{.1\textwidth}

\end{columns}
\vspace{4cm}
\end{frame}


%_____________________________________________________________
\begin{frame}[fragile]{}

\begin{exampleblock}{main.c}
\begin{minted}{c}
int main(void) {
    dht_params_t my_params;
    my_params.pin = GPIO_PIN(PORT_A, 10);
    my_params.type = DHT_PARAM_TYPE;
    my_params.in_mode = DHT_PARAM_PULL;

    dht_t dev;
    if (dht_init(&dev, &my_params) == DHT_OK) {
        printf("DHT sensor connected\n");
    }
}
\end{minted}
\end{exampleblock}

\begin{itemize}

\item Use the \mintinline{c}{struct dht_params_t} to provide the PIN used.

\end{itemize}

\vspace{2cm}
\end{frame}


%_____________________________________________________________
\begin{frame}[fragile]{}

\begin{exampleblock}{}
\begin{minted}[fontsize=\small]{c}
int main(void) {
    int16_t temp, hum;
    if (dht_read(&dev, &temp, &hum) != DHT_OK) {
        printf("Error reading values\n");
    }
    
    char temp_s[10];
    size_t n = fmt_s16_dfp(temp_s, temp, -1);
    temp_s[n] = '\0';
    
    char hum_s[10];
    n = fmt_s16_dfp(hum_s, hum, -1);
    hum_s[n] = '\0';        

    printf("DHT values - temp: %s°C - relative humidity: %s%%\n",
           temp_s, hum_s);
}
\end{minted}
\end{exampleblock}
\vspace{2cm}
\end{frame}

%_____________________________________________________________
\begin{frame}[fragile]{}

\begin{exampleblock}{}
\begin{minted}[fontsize=\small]{c}
int main(void) {
    const int mode = 0;
    const int delay = 5;

    printf("Setting wakeup from mode %d in %d seconds.\n", mode, delay);
    fflush(stdout);

    /* Setting a timer on the RTC */
    struct tm time;
    rtc_get_time(&time);
    time.tm_sec += delay;
    rtc_set_alarm(&time, callback_rtc, "Wakeup alarm");

    /* Enter deep sleep mode */
    pm_set(mode); 
}
\end{minted}
\end{exampleblock}
\vspace{2cm}
\end{frame}



%_____________________________________________________________
\begin{frame}[fragile]{}
\begin{itemize}

\item Setting an alarm to the RTC requires a callback function.

\end{itemize}

\vspace{.5cm}

\begin{exampleblock}{main.c}
\begin{minted}{c}
/**
 * Call-back function invoked when RTC alarm triggers wakeup.
 */
static void callback_rtc(void *arg)
{
    puts(arg);
}
\end{minted}
\end{exampleblock}

\vspace{2cm}
\end{frame}
\end{document}

